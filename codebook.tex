\documentclass[10pt,twocolumn,oneside,a4paper]{article}
\setlength{\columnsep}{10pt}                    %兩欄模式的間距
\setlength{\columnseprule}{0pt}                 %兩欄模式間格線粗細

\usepackage{amsthm}								%定義,例題
\usepackage{amssymb}
\usepackage{fontspec}							%設定字體
\usepackage{color}
\usepackage[x11names]{xcolor}
\usepackage{listings}							%顯示code用的
\usepackage{fancyhdr}							%設定頁首頁尾
\usepackage{graphicx}							%Graphic
\usepackage{enumerate}
\usepackage{titlesec}
\usepackage{amsmath}
\usepackage[CheckSingle, CJKmath]{xeCJK}
 \usepackage{CJKulem}
\usepackage{tikz} % Used for draw Finite Automata

\usepackage{amsmath, courier, listings, fancyhdr, graphicx}
\topmargin=0pt
\headsep=5pt
\textheight=780pt
\footskip=0pt
\voffset=-50pt
\textwidth=545pt
\marginparsep=0pt
\marginparwidth=0pt
\marginparpush=0pt
\oddsidemargin=0pt
\evensidemargin=0pt
\hoffset=-42pt

%\renewcommand\listfigurename{圖目錄}
%\renewcommand\listtablename{表目錄}

%%%%%%%%%%%%%%%%%%%%%%%%%%%%%

\setmainfont{Consolas}
\setmonofont{Consolas}
\XeTeXlinebreaklocale "zh"						%中文自動換行
\XeTeXlinebreakskip = 0pt plus 1pt				%設定段落之間的距離
\setcounter{secnumdepth}{3}						%目錄顯示第三層

%%%%%%%%%%%%%%%%%%%%%%%%%%%%%
\makeatletter
\lst@CCPutMacro\lst@ProcessOther {"2D}{\lst@ttfamily{-{}}{-{}}}
\@empty\z@\@empty
\makeatother
\lstset{										% Code顯示
    language=C++,									% the language of the code
    basicstyle=\footnotesize\ttfamily, 					% the size of the fonts that are used for the code
    %numbers=left,									% where to put the line-numbers
    numberstyle=\footnotesize,					% the size of the fonts that are used for the line-numbers
    stepnumber=1,									% the step between two line-numbers. If it's 1, each line  will be numbered
    numbersep=5pt,									% how far the line-numbers are from the code
    backgroundcolor=\color{white},				% choose the background color. You must add \usepackage{color}
    showspaces=false,								% show spaces adding particular underscores
    showstringspaces=false,						% underline spaces within strings
    showtabs=false,								% show tabs within strings adding particular underscores
    frame=false,										% adds a frame around the code
    tabsize=2,										% sets default tabsize to 2 spaces
    captionpos=b,									% sets the caption-position to bottom
    breaklines=true,								% sets automatic line breaking
    breakatwhitespace=false,						% sets if automatic breaks should only happen at whitespace
    escapeinside={\%*}{*)},						% if you want to add a comment within your code
    morekeywords={*},								% if you want to add more keywords to the set
    keywordstyle=\bfseries\color{Blue1},
    commentstyle=\itshape\color{Red4},
    stringstyle=\itshape\color{Green4},
}

%----------------------------------------------------------------------------------------
%   設定 section 行距
%----------------------------------------------------------------------------------------

\makeatletter
\let\origsection\section
\renewcommand\section{\@ifstar{\starsection}{\nostarsection}}

\newcommand\nostarsection[1]
{\sectionprelude\origsection{#1}\sectionpostlude}

\newcommand\starsection[1]
{\sectionprelude\origsection*{#1}\sectionpostlude}

\newcommand\sectionprelude{%
  \vspace{-10pt} % 改這個數字
}
\newcommand\sectionpostlude{%
  \vspace{-10pt} % 改這個數字
}
\makeatother

% ---------------------------------------------------------------------------------------

\begin{document}
\pagestyle{fancy}
\fancyfoot{}
\fancyhead[L]{OldDriverTrie}
\fancyhead[R]{(\today) \thepage}
\renewcommand{\headrulewidth}{0.4pt}
\renewcommand{\contentsname}{Contents}

\scriptsize
\tableofcontents
\newpage
\section{Basic}
    \subsection{Default\_code}
        \lstinputlisting{basic/default_code.cpp}
    \subsection{vimrc}
        \lstinputlisting{basic/vimrc}
\section{Python}
    \subsection{Syntax}
        \lstinputlisting{python/syntax.py}

\section{Data\_structure}
    \subsection{SegmentTree}
        \lstinputlisting{data_structure/SegmentTree.cpp}
    \subsection{Sparse\_table}
        \lstinputlisting{data_structure/Sparse_table.cpp}
    \subsection{Treap}
        \lstinputlisting{data_structure/Treap.cpp}
    \subsection{Splay\_tree}
        \lstinputlisting{data_structure/Splay_tree.cpp}
    \subsection{Trie}
        \lstinputlisting{data_structure/Trie.cpp}
    \subsection{Persistent\_SegmentTree}
        \lstinputlisting{data_structure/Persistent_segment_tree.cpp}
    \subsection{Lichao\_tree}
        \lstinputlisting{data_structure/Lichao_tree.cpp}
    \subsection{Link-Cut\_tree}
        \lstinputlisting{data_structure/link_cut_tree.cpp}

\section{Flow}
    \subsection{Dinic}
        \lstinputlisting{Flow/dinic.cpp}
    \subsection{Minimum cost maximum flow}
        \lstinputlisting{Flow/mcmf.cpp}

\section{Graph}
    \subsection{Dijkstra}
        \lstinputlisting{graph/dijkstra.cpp}
    \subsection{Kth\_shrtest\_path}
        \lstinputlisting{graph/kth_shortest_path.cpp}
    \subsection{euler\_tour}
        \lstinputlisting{graph/euler_tour.cpp}
    \subsection{Hungarian}
        \lstinputlisting{graph/bipartite graph matching(hungarian).cpp}
    \subsection{2-SAT}
        \lstinputlisting{graph/2-sat.cpp}
    \subsection{SCC}
        \lstinputlisting{graph/scc.cpp}
    \subsection{BCC}
        \lstinputlisting{graph/bcc.cpp}
    \subsection{Tree\_Isomorphism}
        \lstinputlisting{graph/Tree Isomorphism.cpp}

\section{Math}
    \subsection{Bignumber}
        \lstinputlisting{math/Bignumber.cpp}
    \subsection{Exgcd}
        \lstinputlisting{math/exgcd.cpp}
    \subsection{Linear\_sieve}
        \lstinputlisting{math/linear_sieve.cpp}
    \subsection{Linear\_inv}
        \lstinputlisting{math/linear_inv.cpp}
    \subsection{Gaussian\_Elimination(mod)}
        \lstinputlisting{math/Gaussian Elimination(mod).cpp}
    \subsection{Euler\_phi}
        \lstinputlisting{math/euler_phi.cpp}
    \subsection{Chinese remainder theorem}
        \lstinputlisting{math/CRT.cpp}
    \subsection{Miller Rabin}
        \lstinputlisting{math/Miller_Rabin.cpp}
    \subsection{Hamel Basis}
        \lstinputlisting{math/Hamel_basis.cpp}
    \subsection{numbers}
        \begin{itemize}
\item Bernoulli numbers

$B_0-1,B_1^{\pm}=\pm\frac{1}{2},B_2=\frac{1}{6},B_3=0$

$\displaystyle\sum_{j=0}^m\binom{m+1}{j}B_j=0$, EGF is $B(x) = \frac{x}{e^x - 1}=\displaystyle\sum_{n=0}^\infty B_n\frac{x^n}{n!}$.

$S_m(n)=\displaystyle\sum_{k=1}^nk^m=\frac{1}{m+1}\sum_{k=0}^m\binom{m+1}{k}B^{+}_kn^{m+1-k}$

\item Stirling numbers of the second kind
Partitions of $n$ distinct elements into exactly $k$ groups. 

$S(n, k) = S(n - 1, k - 1) + kS(n - 1, k), S(n, 1) = S(n, n) = 1$

$S(n, k) = \frac{1}{k!}\sum_{i=0}^{k}(-1)^{k-i}{k \choose i}i^n$

$x^n     = \sum_{i=0}^{n} S(n, i) (x)_i$

\item Pentagonal number theorem

$\displaystyle\prod_{n=1}^{\infty}(1-x^n)=1+\sum_{k=1}^{\infty}(-1)^k\left(x^{k(3k+1)/2} + x^{k(3k-1)/2}\right)$

\item Catalan numbers

$C^{(k)}_n = \displaystyle \frac{1}{(k - 1)n + 1}\binom{kn}{n}$

$C^{(k)}(x) = 1 + x [C^{(k)}(x)]^k$

\item Eulerian numbers

Number of permutations $\pi \in S_n$ in which exactly $k$ elements are greater than the previous element. $k$ $j$:s s.t. $\pi(j)>\pi(j+1)$, $k+1$ $j$:s s.t. $\pi(j)\geq j$, $k$ $j$:s s.t. $\pi(j)>j$.

$E(n,k) = (n-k)E(n-1,k-1) + (k+1)E(n-1,k)$

$E(n,0) = E(n,n-1) = 1$

$E(n,k) = \sum_{j=0}^k(-1)^j\binom{n+1}{j}(k+1-j)^n$

\end{itemize}
    \subsection{Theorem}
        \begin{itemize}
    \item Cramer's rule
    $$
    \begin{aligned}ax+by=e\\cx+dy=f\end{aligned}
    \Rightarrow
    \begin{aligned}x=\dfrac{ed-bf}{ad-bc}\\y=\dfrac{af-ec}{ad-bc}\end{aligned}
    $$
    
    \item Vandermonde's Identity
    $$
    C(n + m, k) = \sum_{i=0}^k C(n, i)C(m, k - i)
    $$
    
    \item Kirchhoff's Theorem
    
    Denote $L$ be a $n \times n$ matrix as the Laplacian matrix of graph $G$, where $L_{ii} = d(i)$, $L_{ij} = -c$ where $c$ is the number of edge $(i, j)$ in $G$.
    \begin{itemize}
        \itemsep-0.5em
        \item The number of undirected spanning in $G$ is $\lvert \det(\tilde{L}_{11}) \rvert$.
        \item The number of directed spanning tree rooted at $r$ in $G$ is $\lvert \det(\tilde{L}_{rr}) \rvert$.
    \end{itemize}
    
    \item Tutte's Matrix
    
    Let $D$ be a $n \times n$ matrix, where $d_{ij} = x_{ij}$ ($x_{ij}$ is chosen uniformly at random) if $i < j$ and $(i, j) \in E$, otherwise $d_{ij} = -d_{ji}$. $\frac{rank(D)}{2}$ is the maximum matching on $G$.
    
    \item Cayley's Formula
    
    \begin{itemize}
        \itemsep-0.5em
      \item Given a degree sequence $d_1, d_2, \ldots, d_n$ for each \textit{labeled} vertices, there are $\frac{(n - 2)!}{(d_1 - 1)!(d_2 - 1)!\cdots(d_n - 1)!}$ spanning trees.
      \item Let $T_{n, k}$ be the number of \textit{labeled} forests on $n$ vertices with $k$ components, such that vertex $1, 2, \ldots, k$ belong to different components. Then $T_{n, k} = kn^{n - k - 1}$.
    \end{itemize}
    
    \item Erdős–Gallai theorem 
    
    A sequence of nonnegative integers $d_1\ge\cdots\ge d_n$ can be represented as the degree sequence of a finite simple graph on $n$ vertices if and only if $d_1+\cdots+d_n$ is even and $\displaystyle\sum_{i-1}^kd_i\le k(k-1)+\displaystyle\sum_{i=k+1}^n\min(d_i,k)$ holds for every $1\le k\le n$.
    
    \item Gale–Ryser theorem
    
    A pair of sequences of nonnegative integers $a_1\ge\cdots\ge a_n$ and $b_1,\ldots,b_n$ is bigraphic if and only if $\displaystyle\sum_{i=1}^n a_i=\displaystyle\sum_{i=1}^n b_i$ and $\displaystyle\sum_{i=1}^k a_i\le \displaystyle\sum_{i=1}^n\min(b_i,k)$ holds for every $1\le k\le n$.
    
    \item Fulkerson–Chen–Anstee theorem
    
    A sequence $(a_1,b_1),\ldots,(a_n,b_n)$ of nonnegative integer pairs with $a_1\ge\cdots\ge a_n$ is digraphic if and only if $\displaystyle\sum_{i=1}^n a_i=\displaystyle\sum_{i=1}^n b_i$ and $\displaystyle\sum_{i=1}^k a_i\le \displaystyle\sum_{i=1}^k\min(b_i,k-1)+\displaystyle\sum_{i=k+1}^n\min(b_i,k)$ holds for every $1\le k\le n$.
    
    \item Möbius inversion formula
    
    \begin{itemize}
        \itemsep-0.5em
      \item $f(n)=\sum_{d\mid n}g(d)\Leftrightarrow g(n)=\sum_{d\mid n}\mu(d)f(\frac{n}{d})$
      \item $f(n)=\sum_{n\mid d}g(d)\Leftrightarrow g(n)=\sum_{n\mid d}\mu(\frac{d}{n})f(d)$
    \end{itemize}
    
    \item Spherical cap
    
    \begin{itemize}
        \itemsep-0.5em
      \item A portion of a sphere cut off by a plane.
      \item $r$: sphere radius, $a$: radius of the base of the cap, $h$: height of the cap, $\theta$: $\arcsin(a/r)$.
      \item Volume $=\pi h^2(3r-h)/3=\pi h(3a^2+h^2)/6=\pi r^3(2+\cos\theta)(1-\cos\theta)^2/3$.
      \item Area $=2\pi rh=\pi(a^2+h^2)=2\pi r^2(1-\cos\theta)$.
    \end{itemize}
    
    \item Lagrange multiplier
    
    \begin{itemize}
        \itemsep-0.5em
      \item Optimize $f(x_1, \ldots, x_n)$ when $k$ constraints $g_i(x_1, \ldots, x_n)=0$.
      \item Lagrangian function $\mathcal{L}(x_1, \ldots, x_n, \lambda_1, \ldots, \lambda_k) = f(x_1, \ldots, x_n) = \sum^{k}_{i=1}\lambda_i g_i(x_1, \ldots, x_n)$.
      \item The solution corresponding to the original constrained optimization is always a saddle point of the Lagrangian function.
    \end{itemize}
    
    
    \end{itemize}
    \subsection{Generating\_functions}
        \begin{itemize}
    \item Ordinary Generating Function
    $A(x) = \sum_{i\ge 0} a_ix^i$
    \begin{itemize}
        \itemsep-0.5em
        \item $A(rx)             \Rightarrow r^na_n$
        \item $A(x) + B(x)       \Rightarrow a_n + b_n$
        \item $A(x)B(x)          \Rightarrow \sum_{i=0}^{n} a_ib_{n-i}$
        \item $A(x)^k            \Rightarrow \sum_{i_1+i_2+\cdots+i_k=n} a_{i_1}a_{i_2}\ldots a_{i_k}$
        \item $xA(x)'            \Rightarrow na_n$
        \item $\frac{A(x)}{1-x}  \Rightarrow \sum_{i=0}^{n} a_i$
    \end{itemize}
    \item Exponential Generating Function
    $A(x) = \sum_{i\ge 0} \frac{a_i}{i!}x_i$
    \begin{itemize}
        \itemsep-0.5em
        \item $A(x) + B(x)       \Rightarrow a_n + b_n$
        \item $A^{(k)}(x)        \Rightarrow a_{n+k}$
        \item $A(x)B(x)          \Rightarrow \sum_{i=0}^{n} \binom{n}{i}a_ib_{n-i}$
        \item $A(x)^k            \Rightarrow \sum_{i_1+i_2+\cdots+i_k=n} \binom{n}{i_1, i_2, \ldots, i_k}a_{i_1}a_{i_2}\ldots a_{i_k}$
        \item $xA(x)             \Rightarrow na_n$
    \end{itemize}
    \item Special Generating Function
    \begin{itemize}
        \itemsep-0.5em
        \item $(1+x)^n           = \sum_{i\ge 0} \binom{n}{i}x^i$
        \item $\frac{1}{(1-x)^n} = \sum_{i\ge 0} \binom{i}{n-1}x^i$
    \end{itemize}
    \end{itemize}

\section{Geometry}
\subsection{Geomerty\_Default}
        \lstinputlisting{geometry/geometry_default.cpp}
    \subsection{Convexhull}
        \lstinputlisting{geometry/Convex_hull.cpp}
    \subsection{Closest\_pair}
        \lstinputlisting{geometry/closest pair.cpp}
    \subsection{Farthest\_pair}
        \lstinputlisting{geometry/farthest_pair.cpp}
    \subsection{Smallest\_enclosing\_circle}
        \lstinputlisting{geometry/smallest_enclosing_circle.cpp}
    \subsection{Rectangles\_area}
        \lstinputlisting{geometry/rectangles area.cpp}

\section{String}
    \subsection{KMP}
        \lstinputlisting{string/KMP.cpp}
    \subsection{Z\_algorithm}
        \lstinputlisting{string/Z_algorithm.cpp}
    \subsection{Smallest\_rotation}
        \lstinputlisting{string/smallest_rotation.cpp}
    \subsection{Manacher}
        \lstinputlisting{string/manacher.cpp}
    \subsection{AC\_automaton}
        \lstinputlisting{string/AC_automaton.cpp}
    \subsection{Suffix\_Array}
        \lstinputlisting{string/Suffix_Array.cpp}

\section{Others}
    \subsection{CDQ}
        \lstinputlisting{others/CDQ.cpp}
    \subsection{Digital\_dp}
        \lstinputlisting{others/Digital_dp.cpp}
    \subsection{Matrix\_fpow}
        \lstinputlisting{others/matrix_fpow.cpp}
    \subsection{Mo's\_algorithm}
        \lstinputlisting{others/Mo's_algorithm.cpp}
    \subsection{time\_segment\_tree}
        \lstinputlisting{others/time_segment_tree.cpp}


\end{document}